% ****************************************************************************************
% **********************    SUCESIONES Y SERIES INFINITAS     ****************************
% ****************************************************************************************


% =======================================================
% =======         HEADER FOR DOCUMENT        ============
% =======================================================
    
    % *********  SPECIFIC FOR THIS BOOK  ********
    \def\ProjectAuthorLink{https://github.com/CompilandoConocimiento}
    \def\ProjectNameLink{\ProjectAuthorLink/LibroSucesionesYSeries}    
    

    % *********   DOCUMENT ITSELF   **************
    \documentclass[12pt, fleqn]{report}                             %Type of doc and size of font and left equations
    \usepackage[margin=1.2in]{geometry}                             %Margins and Geometry pacakge
    \usepackage{ifthen}                                             %Allow simple programming using if - then
    \usepackage[hidelinks]{hyperref}                                %Allow to create hiperlinks and Fuck Firefox
    \usepackage{pdfpages}                                           %Allow us 'import' PDF's
    \hypersetup{pageanchor=false}                                   %Solve 'double page 1' warnings in build :v
    \setlength{\parindent}{0pt}                                     %Eliminate ugly indentation
    \author{Oscar Andrés Rosas}                                     %Who I am

    % *********   LANGUAJE    *****************
    \usepackage[spanish]{babel}                                     %Please allow me to type in spanish
    \usepackage[utf8]{inputenc}                                     %Lets use UFT-8
    \usepackage[T1]{fontenc}                                        %Allow for better font support
    \usepackage{textcmds}                                           %Allow us to use quoutes
    \usepackage{changepage}                                         %Allow us to use identate paragraphs
    \usepackage{anyfontsize}                                        %All the sizes for fonts wiiiii!

    % *********   MATH AND HIS STYLE  *********
    \usepackage{ntheorem, amsmath, amssymb, amsfonts}               %All fucking math, I want all!
    \usepackage{mathrsfs, mathtools, empheq}                        %All fucking math, I want all!
    \usepackage{cancel}                                             %Negate symbol
    \usepackage{centernot}                                          %Allow me to negate a symbol
    \decimalpoint                                                   %Use decimal point

    % *********   GRAPHICS AND IMAGES *********
    \usepackage{graphicx}                                           %Allow to create graphics
    \usepackage{float}                                              %For images
    \usepackage{wrapfig}                                            %Allow to create images
    \graphicspath{ {Graphics/} }                                    %Where are the images :D

    % *********   LISTS AND TABLES ***********
    \usepackage{listings, listingsutf8}                             %We will be using code here
    \usepackage[inline]{enumitem}                                   %We will need to enumarate
    \usepackage{tasks}                                              %Horizontal lists
    \usepackage{longtable}                                          %Lets make tables awesome
    \usepackage{booktabs}                                           %Lets make tables awesome
    \usepackage{tabularx}                                           %Lets make tables awesome
    \usepackage{multirow}                                           %Lets make tables awesome
    \usepackage{multicol}                                           %Create multicolumns

    % *********   REMOVE SOME ERRORS **********
    \hbadness=10000                                                 %Ignore \vbox and \hbox warings
    \hfuzz=\maxdimen\newdimen\hfuzz                                 %Ignore \vbox and \hbox warings

    % *********   HEADERS AND FOOTERS ********
    \usepackage{fancyhdr}                                           %Lets make awesome headers/footers
    \pagestyle{fancy}                                               %Lets make awesome headers/footers
    \setlength{\headheight}{16pt}                                   %Top line
    \setlength{\parskip}{0.5em}                                     %Top line
    \renewcommand{\footrulewidth}{0.5pt}                            %Bottom line

    \lhead {                                                        %Left Header
        \hyperlink{chapter.\arabic{chapter}}                        %Make a link to the current chapter
        {\normalsize{\textsc{\nouppercase{\leftmark}}}}             %And fot it put the name
    }

    \rhead {                                                        %Right Header
        \hyperlink{section.\arabic{chapter}.\arabic{section}}       %Make a link to the current chapter
            {\footnotesize{\textsc{\nouppercase{\rightmark}}}}      %And fot it put the name
    }

    \rfoot{\textsc{\small{\hyperref[sec:Index]{Ve al Índice}}}}     %This will always be a footer  

    \fancyfoot[L]{                                                  %Algoritm for a changing footer
        \ifthenelse{\isodd{\value{page}}}                           %IF ODD PAGE:
            {\href{https://SoyOscarRH.github.io/}                   %DO THIS:
                {\footnotesize                                      %Send the page
                    {\textsc{Oscar Andrés Rosas}}}}                 %Send the page
            {\href{https://compilandoconocimiento.com}              %ELSE DO THIS: 
                {\footnotesize                                      %Send the author
                    {\textsc{Compilando Conocimiento}}}}            %Send the author
    }
    
    
% =======================================================
% ===================   COMMANDS    =====================
% =======================================================

    % =========================================
    % =======   NEW ENVIRONMENTS   ============
    % =========================================
    \newenvironment{Indentation}[1][0.75em]                         %Use: \begin{Inde...}[Num]...\end{Inde...}
        {\begin{adjustwidth}{#1}{}}                                 %If you dont put nothing i will use 0.75 em
        {\end{adjustwidth}}                                         %This indentate a paragraph
    
    \newenvironment{SmallIndentation}[1][0.75em]                    %Use: The same that we upper one, just 
        {\begin{adjustwidth}{#1}{}\begin{footnotesize}}             %footnotesize size of letter by default
        {\end{footnotesize}\end{adjustwidth}}                       %that's it
    
    \def \Eq {equation}                                             %Stupid Visual studio error
    \newenvironment{MultiLineEquation}[1]                           %Use: To create MultiLine equations
        {\begin{\Eq}\begin{alignedat}{#1}}                          %Use: \begin{Multi..}{Num. de Columnas}
        {\end{alignedat}\end{\Eq}}                                  %And.. that's it!
    
    \newenvironment{MultiLineEquation*}[1]                          %Use: To create MultiLine equations
        {\begin{\Eq*}\begin{alignedat}{#1}}                         %Use: \begin{Multi..}{Num. de Columnas}
        {\end{alignedat}\end{\Eq*}}                                 %And.. that's it!
    

    % =========================================
    % == GENERAL TEXT & SYMBOLS ENVIRONMENTS ==
    % =========================================
    
    % =====  TEXT  ======================
    \newcommand \Quote              {\qq}                           %Use: \Quote to use quotes
    \newcommand \Over               {\overline}                     %Use: \Bar to use just for short
    \newcommand \ForceNewLine       {$\Space$\\}                    %Use it in theorems for example
    \newcommand \ForceColumnBreak   {\vfill\null\columnbreak}       %Use only in multicols

    % =====  SPACES  ====================
    \DeclareMathOperator \Space     {\quad}                         %Use: \Space for a cool mega space
    \DeclareMathOperator \MegaSpace {\quad \quad}                   %Use: \MegaSpace for a cool mega mega space
    \DeclareMathOperator \MiniSpace {\;}                            %Use: \Space for a cool mini space
    
    % =====  MATH TEXT  =================
    \newcommand \Such           {\MiniSpace | \MiniSpace}           %Use: \Such like in sets
    \newcommand \Also           {\MiniSpace \text{y} \MiniSpace}    %Use: \Also so it's look cool
    \newcommand \Remember[1]    {\Space\text{\scriptsize{#1}}}      %Use: \Remember so it's look cool
    
    % =====  THEOREMS: IN SPANISH :0  ===
    \newtheorem{Theorem}        {Teorema}[section]                  %Use: \begin{Theorem}[Name]\label{Nombre}...
    \newtheorem{Corollary}      {Colorario}[Theorem]                %Use: \begin{Corollary}[Name]\label{Nombre}...
    \newtheorem{Lemma}[Theorem] {Lemma}                             %Use: \begin{Lemma}[Name]\label{Nombre}...
    \newtheorem{Definition}     {Definición}[section]               %Use: \begin{Definition}[Name]\label{Nombre}...
    \theoremstyle{break}                                            %THEOREMS START 1 SPACE AFTER Fuck!

    % =====  LOGIC  =====================
    \newcommand \lIff    {\leftrightarrow}                          %Use: \lIff for logic iff
    \newcommand \lEqual  {\MiniSpace \Leftrightarrow \MiniSpace}    %Use: \lEqual for a logic double arrow
    \newcommand \lInfire {\MiniSpace \Rightarrow \MiniSpace}        %Use: \lInfire for a logic infire
    \newcommand \lLongTo {\longrightarrow}                          %Use: \lLongTo for a long arrow

    % =====  FAMOUS SETS  ===============
    \DeclareMathOperator \Naturals     {\mathbb{N}}                 %Use: \Naturals por Notation
    \DeclareMathOperator \Primes       {\mathbb{P}}                 %Use: \Primes por Notation
    \DeclareMathOperator \Integers     {\mathbb{Z}}                 %Use: \Integers por Notation
    \DeclareMathOperator \Racionals    {\mathbb{Q}}                 %Use: \Racionals por Notation
    \DeclareMathOperator \Reals        {\mathbb{R}}                 %Use: \Reals por Notation
    \DeclareMathOperator \Complexs     {\mathbb{C}}                 %Use: \Complex por Notation
    \DeclareMathOperator \GenericField {\mathbb{F}}                 %Use: \GenericField por Notation
    \DeclareMathOperator \VectorSet    {\mathbb{V}}                 %Use: \VectorSet por Notation
    \DeclareMathOperator \SubVectorSet {\mathbb{W}}                 %Use: \SubVectorSet por Notation
    \DeclareMathOperator \Polynomials  {\mathbb{P}}                 %Use: \Polynomials por Notation
    \DeclareMathOperator \VectorSpace  {\VectorSet_{\GenericField}} %Use: \VectorSpace por Notation
    \DeclareMathOperator \LinealTransformation {\mathcal{T}}        %Use: \LinealTransformation for a cool T
    \DeclareMathOperator \LinTrans      {\mathcal{T}}               %Use: \LinTrans for a cool T
    \DeclareMathOperator \Laplace       {\mathcal{L}}               %Use: \LinTrans for a cool T

    % =====  CONTAINERS   ===============
    \newcommand{\Set}[1]            {\left\{ \; #1 \; \right\}}     %Use: \Set {Info} for INTELLIGENT space 
    \newcommand{\bigSet}[1]         {\big\{  \; #1 \; \big\}}       %Use: \bigSet  {Info} for space 
    \newcommand{\BigSet}[1]         {\Big\{  \; #1 \; \Big\}}       %Use: \BigSet  {Info} for space 
    \newcommand{\biggSet}[1]        {\bigg\{ \; #1 \; \bigg\}}      %Use: \biggSet {Info} for space 
    \newcommand{\BiggSet}[1]        {\Bigg\{ \; #1 \; \Bigg\}}      %Use: \BiggSet {Info} for space 
        
    \newcommand{\Wrap}[1]           {\left( #1 \right)}             %Use: \Wrap {Info} for INTELLIGENT space
    \newcommand{\bigWrap}[1]        {\big( \; #1 \; \big)}          %Use: \bigBrackets  {Info} for space 
    \newcommand{\BigWrap}[1]        {\Big( \; #1 \; \Big)}          %Use: \BigBrackets  {Info} for space 
    \newcommand{\biggWrap}[1]       {\bigg( \; #1 \; \bigg)}        %Use: \biggBrackets {Info} for space 
    \newcommand{\BiggWrap}[1]       {\Bigg( \; #1 \; \Bigg)}        %Use: \BiggBrackets {Info} for space 

    \newcommand{\Brackets}[1]       {\left[ #1 \right]}             %Use: \Brackets {Info} for INTELLIGENT space
    \newcommand{\bigBrackets}[1]    {\big[ \; #1 \; \big]}          %Use: \bigBrackets  {Info} for space 
    \newcommand{\BigBrackets}[1]    {\Big[ \; #1 \; \Big]}          %Use: \BigBrackets  {Info} for space 
    \newcommand{\biggBrackets}[1]   {\bigg[ \; #1 \; \bigg]}        %Use: \biggBrackets {Info} for space 
    \newcommand{\BiggBrackets}[1]   {\Bigg[ \; #1 \; \Bigg]}        %Use: \BiggBrackets {Info} for space 

    \newcommand{\Generate}[1]   {\left\langle #1 \right\rangle}     %Use: \Generate {Info} <>
    \newcommand{\Floor}[1]      {\left \lfloor #1 \right \rfloor}   %Use: \Floor {Info} for floor 
    \newcommand{\Ceil}[1]       {\left \lceil #1 \right \rceil }    %Use: \Ceil {Info} for ceil
    
    % =====  BETTERS MATH COMMANDS   =====
    \newcommand{\pfrac}[2]      {\Wrap{\dfrac{#1}{#2}}}             %Use: Put fractions in parentesis

    % =========================================
    % ====   LINEAL ALGEBRA & VECTORS    ======
    % =========================================

    % ===== UNIT VECTORS  ================
    \newcommand{\hati}      {\hat{\imath}}                           %Use: \hati for unit vector    
    \newcommand{\hatj}      {\hat{\jmath}}                           %Use: \hatj for unit vector    
    \newcommand{\hatk}      {\hat{k}}                                %Use: \hatk for unit vector

    % ===== MAGNITUDE  ===================
    \newcommand{\abs}[1]    {\left\lvert #1 \right\lvert}           %Use: \abs{expression} for |x|
    \newcommand{\Abs}[1]    {\left\lVert #1 \right\lVert}           %Use: \Abs{expression} for ||x||
    \newcommand{\Mag}[1]    {\left| #1 \right|}                     %Use: \Mag {Info} 
    
    \newcommand{\bVec}[1]   {\mathbf{#1}}                           %Use for bold type of vector
    \newcommand{\lVec}[1]   {\overrightarrow{#1}}                   %Use for a long arrow over a vector
    \newcommand{\uVec}[1]   {\mathbf{\hat{#1}}}                     %Use: Unitary Vector Example: $\uVec{i}

    % ===== FN LINEAL TRANSFORMATION  ====
    \newcommand{\FnLinTrans}[1]{\mathcal{T}\Wrap{#1}}               %Use: \FnLinTrans for a cool T
    \newcommand{\VecLinTrans}[1]{\mathcal{T}\pVector{#1}}           %Use: \LinTrans for a cool T
    \newcommand{\FnLinealTransformation}[1]{\mathcal{T}\Wrap{#1}}   %Use: \FnLinealTransformation

    % ===== ALL FOR DOT PRODUCT  =========
    \makeatletter                                                   %WTF! IS THIS
    \newcommand*\dotP{\mathpalette\dotP@{.5}}                       %Use: \dotP for dot product
    \newcommand*\dotP@[2] {\mathbin {                               %WTF! IS THIS            
        \vcenter{\hbox{\scalebox{#2}{$\m@th#1\bullet$}}}}           %WTF! IS THIS
    }                                                               %WTF! IS THIS
    \makeatother                                                    %WTF! IS THIS

    % === WRAPPERS FOR COLUMN VECTOR ===
    \newcommand{\pVector}[1]                                        %Use: \pVector {Matrix Notation} use parentesis
        { \ensuremath{\begin{pmatrix}#1\end{pmatrix}} }             %Example: \pVector{a\\b\\c} or \pVector{a&b&c} 
    \newcommand{\lVector}[1]                                        %Use: \lVector {Matrix Notation} use a abs 
        { \ensuremath{\begin{vmatrix}#1\end{vmatrix}} }             %Example: \lVector{a\\b\\c} or \lVector{a&b&c} 
    \newcommand{\bVector}[1]                                        %Use: \bVector {Matrix Notation} use a brackets 
        { \ensuremath{\begin{bmatrix}#1\end{bmatrix}} }             %Example: \bVector{a\\b\\c} or \bVector{a&b&c} 
    \newcommand{\Vector}[1]                                         %Use: \Vector {Matrix Notation} no parentesis
        { \ensuremath{\begin{matrix}#1\end{matrix}} }               %Example: \Vector{a\\b\\c} or \Vector{a&b&c}

    % === MAKE MATRIX BETTER  =========
    \makeatletter                                                   %Example: \begin{matrix}[cc|c]
    \renewcommand*\env@matrix[1][*\c@MaxMatrixCols c] {             %WTF! IS THIS
        \hskip -\arraycolsep                                        %WTF! IS THIS
        \let\@ifnextchar\new@ifnextchar                             %WTF! IS THIS
        \array{#1}                                                  %WTF! IS THIS
    }                                                               %WTF! IS THIS
    \makeatother                                                    %WTF! IS THIS

    % =========================================
    % =======   FAMOUS FUNCTIONS   ============
    % =========================================

    % == TRIGONOMETRIC FUNCTIONS  ====
    \newcommand{\Cos}[1] {\cos\Wrap{#1}}                            %Simple wrappers
    \newcommand{\Sin}[1] {\sin\Wrap{#1}}                            %Simple wrappers
    \newcommand{\Tan}[1] {tan\Wrap{#1}}                             %Simple wrappers
    
    \newcommand{\Sec}[1] {sec\Wrap{#1}}                             %Simple wrappers
    \newcommand{\Csc}[1] {csc\Wrap{#1}}                             %Simple wrappers
    \newcommand{\Cot}[1] {cot\Wrap{#1}}                             %Simple wrappers

    % === COMPLEX ANALYSIS TRIG ======
    \newcommand \Cis[1]  {\Cos{#1} + i \Sin{#1}}                    %Use: \Cis for cos(x) + i sin(x)
    \newcommand \pCis[1] {\Wrap{\Cis{#1}}}                          %Use: \pCis for the same with parantesis
    \newcommand \bCis[1] {\Brackets{\Cis{#1}}}                      %Use: \bCis for the same with Brackets


    % =========================================
    % ===========     CALCULUS     ============
    % =========================================

    % ====== TRANSFORMS =============
    \newcommand{\FourierT}[1]   {\mathscr{F} \left\{ #1 \right\} }  %Use: \FourierT {Funtion}
    \newcommand{\InvFourierT}[1]{\mathscr{F}^{-1}\left\{#1\right\}} %Use: \InvFourierT {Funtion}

    % ====== DERIVATIVES ============
    \newcommand \MiniDerivate[1][x]   {\dfrac{d}{d #1}}             %Use: \MiniDerivate[var] for simple use [var]
    \newcommand \Derivate[2]          {\dfrac{d \; #1}{d #2}}       %Use: \Derivate [f(x)][x]
    \newcommand \MiniUpperDerivate[2] {\dfrac{d^{#2}}{d#1^{#2}}}    %Mini Derivate High Orden Derivate -- [x][pow]
    \newcommand \UpperDerivate[3] {\dfrac{d^{#3} \; #1}{d#2^{#3}}}  %Complete High Orden Derivate -- [f(x)][x][pow]
    
    \newcommand \MiniPartial[1][x] {\dfrac{\partial}{\partial #1}}  %Use: \MiniDerivate for simple use [var]
    \newcommand \Partial[2] {\dfrac{\partial \; #1}{\partial #2}}   %Complete Partial Derivate -- [f(x)][x]
    \newcommand \MiniUpperPartial[2]                                %Mini Derivate High Orden Derivate -- [x][pow] 
        {\dfrac{\partial^{#2}}{\partial #1^{#2}}}                   %Mini Derivate High Orden Derivate
    \newcommand \UpperPartial[3]                                    %Complete High Orden Derivate -- [f(x)][x][pow]
        {\dfrac{\partial^{#3} \; #1}{\partial#2^{#3}}}              %Use: \UpperDerivate for simple use

    \DeclareMathOperator \Evaluate  {\Big|}                         %Use: \Evaluate por Notation

    % ====== INTEGRALS ============
    \newcommand{\inftyInt} {\int_{-\infty}^{\infty}}                %Use: \inftyInt for simple integrants
    
        
% =======================================================
% ===========      COLOR: MATERIAL DESIGN     ===========
% =======================================================

    % =====  COLORS ==================
    \definecolor{RedMD}{HTML}{F44336}                               %Use: Color :D        
    \definecolor{Red100MD}{HTML}{FFCDD2}                            %Use: Color :D        
    \definecolor{Red200MD}{HTML}{EF9A9A}                            %Use: Color :D        
    \definecolor{Red300MD}{HTML}{E57373}                            %Use: Color :D        
    \definecolor{Red700MD}{HTML}{D32F2F}                            %Use: Color :D 

    \definecolor{PurpleMD}{HTML}{9C27B0}                            %Use: Color :D        
    \definecolor{Purple100MD}{HTML}{E1BEE7}                         %Use: Color :D        
    \definecolor{Purple200MD}{HTML}{EF9A9A}                         %Use: Color :D        
    \definecolor{Purple300MD}{HTML}{BA68C8}                         %Use: Color :D        
    \definecolor{Purple700MD}{HTML}{7B1FA2}                         %Use: Color :D 

    \definecolor{IndigoMD}{HTML}{3F51B5}                            %Use: Color :D        
    \definecolor{Indigo100MD}{HTML}{C5CAE9}                         %Use: Color :D        
    \definecolor{Indigo200MD}{HTML}{9FA8DA}                         %Use: Color :D        
    \definecolor{Indigo300MD}{HTML}{7986CB}                         %Use: Color :D        
    \definecolor{Indigo700MD}{HTML}{303F9F}                         %Use: Color :D 

    \definecolor{BlueMD}{HTML}{2196F3}                              %Use: Color :D        
    \definecolor{Blue100MD}{HTML}{BBDEFB}                           %Use: Color :D        
    \definecolor{Blue200MD}{HTML}{90CAF9}                           %Use: Color :D        
    \definecolor{Blue300MD}{HTML}{64B5F6}                           %Use: Color :D        
    \definecolor{Blue700MD}{HTML}{1976D2}                           %Use: Color :D        
    \definecolor{Blue900MD}{HTML}{0D47A1}                           %Use: Color :D  

    \definecolor{CyanMD}{HTML}{00BCD4}                              %Use: Color :D        
    \definecolor{Cyan100MD}{HTML}{B2EBF2}                           %Use: Color :D        
    \definecolor{Cyan200MD}{HTML}{80DEEA}                           %Use: Color :D        
    \definecolor{Cyan300MD}{HTML}{4DD0E1}                           %Use: Color :D        
    \definecolor{Cyan700MD}{HTML}{0097A7}                           %Use: Color :D        
    \definecolor{Cyan900MD}{HTML}{006064}                           %Use: Color :D 

    \definecolor{TealMD}{HTML}{009688}                              %Use: Color :D        
    \definecolor{Teal100MD}{HTML}{B2DFDB}                           %Use: Color :D        
    \definecolor{Teal200MD}{HTML}{80CBC4}                           %Use: Color :D        
    \definecolor{Teal300MD}{HTML}{4DB6AC}                           %Use: Color :D        
    \definecolor{Teal700MD}{HTML}{00796B}                           %Use: Color :D        
    \definecolor{Teal900MD}{HTML}{004D40}                           %Use: Color :D 

    \definecolor{GreenMD}{HTML}{4CAF50}                             %Use: Color :D        
    \definecolor{Green100MD}{HTML}{C8E6C9}                          %Use: Color :D        
    \definecolor{Green200MD}{HTML}{A5D6A7}                          %Use: Color :D        
    \definecolor{Green300MD}{HTML}{81C784}                          %Use: Color :D        
    \definecolor{Green700MD}{HTML}{388E3C}                          %Use: Color :D        
    \definecolor{Green900MD}{HTML}{1B5E20}                          %Use: Color :D

    \definecolor{AmberMD}{HTML}{FFC107}                             %Use: Color :D        
    \definecolor{Amber100MD}{HTML}{FFECB3}                          %Use: Color :D        
    \definecolor{Amber200MD}{HTML}{FFE082}                          %Use: Color :D        
    \definecolor{Amber300MD}{HTML}{FFD54F}                          %Use: Color :D        
    \definecolor{Amber700MD}{HTML}{FFA000}                          %Use: Color :D        
    \definecolor{Amber900MD}{HTML}{FF6F00}                          %Use: Color :D

    \definecolor{BlueGreyMD}{HTML}{607D8B}                          %Use: Color :D        
    \definecolor{BlueGrey100MD}{HTML}{CFD8DC}                       %Use: Color :D        
    \definecolor{BlueGrey200MD}{HTML}{B0BEC5}                       %Use: Color :D        
    \definecolor{BlueGrey300MD}{HTML}{90A4AE}                       %Use: Color :D        
    \definecolor{BlueGrey700MD}{HTML}{455A64}                       %Use: Color :D        
    \definecolor{BlueGrey900MD}{HTML}{263238}                       %Use: Color :D        

    \definecolor{DeepPurpleMD}{HTML}{673AB7}                        %Use: Color :D

    % =====  ENVIRONMENT ==============
    \newcommand{\Color}[2]{\textcolor{#1}{#2}}                      %Simple color environment
    \newenvironment{ColorText}[1]                                   %Use: \begin{ColorText}
        { \leavevmode\color{#1}\ignorespaces }                      %That's is!


% =======================================================
% ===========           CODE EDITING          ===========
% =======================================================

    % =====  CODE EDITOR =============
    \lstdefinestyle{CompilandoStyle} {                              %This is Code Style
        backgroundcolor     = \color{BlueGrey900MD},                %Background Color  
        basicstyle          = \tiny\color{white},                   %Style of text
        commentstyle        = \color{BlueGrey200MD},                %Comment style
        stringstyle         = \color{Green300MD},                   %String style
        keywordstyle        = \color{Blue300MD},                    %keywords style
        numberstyle         = \tiny\color{TealMD},                  %Size of a number
        frame               = shadowbox,                            %Adds a frame around the code
        breakatwhitespace   = true,                                 %Style   
        breaklines          = true,                                 %Style   
        showstringspaces    = false,                                %Hate those spaces                  
        breaklines          = true,                                 %Style                   
        keepspaces          = true,                                 %Style                   
        numbers             = left,                                 %Style                   
        numbersep           = 10pt,                                 %Style 
        xleftmargin         = \parindent,                           %Style 
        tabsize             = 4,                                    %Style
        inputencoding       = utf8/latin1                           %Allow me to use special chars
    }

    % =====  CODE EDITOR =============
    \lstdefinestyle{CompilandoStylePurity} {                        %This is Code Style
        backgroundcolor     = \color{white},                        %Background Color  
        basicstyle          = \tiny\color{BlueGrey900MD},           %Style of text
        commentstyle        = \color{Green300MD},                   %Comment style
        stringstyle         = \color{Teal700MD},                    %String style
        keywordstyle        = \color{Blue700MD},                    %keywords style
        numberstyle         = \tiny\color{TealMD},                  %Size of a number
        frame               = none,                                 %Adds a frame around the code
        breakatwhitespace   = true,                                 %Style   
        breaklines          = true,                                 %Style   
        showstringspaces    = false,                                %Hate those spaces                  
        breaklines          = true,                                 %Style                   
        keepspaces          = true,                                 %Style                   
        numbers             = left,                                 %Style                   
        numbersep           = 11pt,                                 %Style 
        xleftmargin         = \parindent,                           %Style 
        tabsize             = 4,                                    %Style
        inputencoding       = utf8/latin1                           %Allow me to use special chars
    }
 
    \lstset{style = CompilandoStyle}                                %Use this style



% =====================================================
% ============        COVER PAGE       ================
% =====================================================
\begin{document}
\begin{titlepage}
    
    % ============ TITLE PAGE STYLE  ================
    \definecolor{TitlePageColor}{cmyk}{1,.60,0,.40}                 %Simple colors
    \definecolor{ColorSubtext}{cmyk}{1,.50,0,.10}                   %Simple colors
    \newgeometry{left=0.20\textwidth}                               %Defines an Offset
    \pagecolor{TitlePageColor}                                      %Make it this Color to page
    \color{white}                                                   %General things should be white

    % ===== MAKE SOME SPACE =========
    \vspace                                                         %Give some space
    \baselineskip                                                   %But we need this to up command

    % ============ NAME OF THE PROJECT  ============
    \makebox[0pt][l]{\rule{1.3\textwidth}{3pt}}                     %Make a cool line
    
    \href{https://compilandoconocimiento.com}                       %Link to project
    {\textbf{\textsc{\Huge Compilando Conocimiento}}}\\[2.7cm]      %Name of project   

    % ============ NAME OF THE BOOK  ===============
    \href{\ProjectNameLink}                                         %Link to Author
    {\fontsize{50}{65}                                              %Size of the book
        \selectfont \textbf{Sucesiones y Series}}\\[0.5cm]          %Name of the book
    \textcolor{ColorSubtext}                                        %Color or the topic
        {\textsc{\LARGE Cálculo y Análisis}}                        %Name of the general theme
    
    \vfill                                                          %Fill the space
    
    % ============ NAME OF THE AUTHOR  =============
    \href{https://compilandoconocimiento.com/nosotros}              %Link to Author
    {\LARGE \textsf{Oscar Andrés Rosas Hernandez}}                  %Author


    % ===== MAKE SOME SPACE =========
    \vspace                                                         %Give some space
    \baselineskip                                                   %But we need this to up command
    
    {\large \textsf{Julio 2018}}                                    %Date

\end{titlepage}


% =====================================================
% ==========      RESTORE TO DOCUMENT      ============
% =====================================================
\restoregeometry                                                    %Restores the geometry
\nopagecolor                                                        %Use to restore the color to white




% =====================================================
% ========                INDICE              =========
% =====================================================
\tableofcontents{}
\label{sec:Index}

\clearpage





% ////////////////////////////////////////////////////////////////////////////////////////////////////////////////////
% ///////////////////////           ¿QUE ES LO QUE ESTOY LEYENDO?           //////////////////////////////////////////
% ////////////////////////////////////////////////////////////////////////////////////////////////////////////////////
\section{¿Qué es lo que estoy leyendo?}
    
    Hola... ¡Hey! Seguramente te estarás preguntando
    ¿Qué demonios estoy leyendo?

    Bueno, este pequeño texto intenta darle solución a esa pregunta, la respuesta mas inmediata es
    que este texto (o compilado como nos gusta decirle) es una recopilación de teoremas, ideas
    y conceptos importantes que aprendí a lo largo del tiempo sobre este tema.

    De manera regular estarémos actualizando estos textos con todo aquello nuevo que aprenda intentando
    profundizar en todos estos temas y cerrar posibles dudas en estas páginas, así que siempre mantente
    alerta de tener la última versión, esta siempre esta en \href{http://www.CompilandoConocimiento.com}
    {\underline{CompilandoConocimiento.com}} 

    Este Compilado intenta ser lo más estricto posible, aunque somos humanos y es posible (e incluso probable) que
    cometamos pequeños errores de vez en cuando.

    Estos textos están creados como una base con la que tu puedas leer rápidamente todo lo que hemos aprendido
    a lo largo del tiempo, aprender los conceptos más importantes y que usándo esto tu puedas profundizar
    más en la maravilla que es aprender más sobre este maravilloso mundo.

    Este texto esta publicado bajo la GPL, por lo tanto es software libre y tu tienes el control total sobre
    el, puedes descargar este texto, puedes ver su código fuente, puedes modificarlo y puedes distribuir este
    texto y sus versiones modificadas, puedes acceder a todo lo que necesitas 
    \href{http://www.github.com/CompilandoConocimiento/LibroCalculoDiferencialEIntegral}
    {\underline{en el Repositorio del Libro de Cálculo Diferencial e Integral}}. 

    Cualquier pregunta, comentario o si quieres contactar con nosotros no dudes en escribir al email del proyecto:
    CompilandoConocimiento@gmail.com

    Espero que tomes estas páginas como un regalo, creado por seres imperfectos pero con muchos ánimos de hacer
    del mundo un lugar mejor, ahora si, abróchate los cinturones que esto acaba de empezar.

    \begin{flushright}
        Compilar es Compartir
    \end{flushright}



% ======================================================================================
% ==================================     SERIES    =====================================
% ======================================================================================
\chapter{Tipos de Series}
    \clearpage

    % =====================================================
    % ========         SERIES GEOMETRICAS           =======
    % =====================================================
    \section{Series Geométricas}

        Una series se dice que es geométrica si es que si divides dos términos
        consecutivos siempre obtendrás la MISMA CONSTANTE.

        Son series del estilo $a + ar + ar^2 + ar^3 + \cdots$, podemos
        generalizarlas como:

        \begin{equation}
            \sum_{n=1}^{\infty} ar^{n-1} = \sum_{n=0}^{\infty} ar^n 
        \end{equation}


        \textbf{Recuerda:}
        Podemos saber facilmente si converge o no, solo basta con que $|r| < 1$ 
        para estar seguros de que converge, donde podemos encontrar a que convege
        también muy fácil como:

        \begin{equation}
            \sum_{n=1}^{\infty} ar^{n-1} = \frac{a}{1-r}
        \end{equation}

        De no ser así, es decir, si $|r| \geq 1$ podemos estar seguros de que diverge.


        % ==========================
        % ====    EJEMPLO  =========
        % ==========================
        \subsubsection{Ejemplo 1}
            Un ejercicio muy sencillo es ver a que converge la siguiente sucesión:
            \begin{equation*}
                5 - \frac{10}{3} + \frac{20}{9} - \frac{40}{27} + \cdots
            \end{equation*}

            Podemos encontrar la respuesta facilmente porque vemos que
            $r=-\frac{2}{3}$ y como $|r|<1$ la Suma es:
            \begin{equation*}
                \frac{5}{1-\frac{2}{3}} = \frac{5}{\frac{5}{3}} = 3
            \end{equation*}



    % =====================================================
    % ========              SERIES - P            =========
    % =====================================================
    \clearpage
    \section{Series P: La Madre de todas las Armónicas}
        Para empezar hay que recordar que hay una serie muy famosa que se conoce como
        la Serie Armónica:
        \begin{equation}
            \sum_{n=1}^{\infty} \frac{1}{n} = 1 + \frac{1}{2}  + \frac{1}{3} + \cdots
        \end{equation}

        Podemos entonces hablar de las Series P, que es una generalización de las series
        armonicas, de la forma:

        \begin{equation}
            \sum_{n=1}^{\infty} \frac{1}{n^P}
        \end{equation}

        \textbf{Recuerda:}
        \begin{itemize}
            \item Cuando $p\leq1$ es la serie armónica (La cual diverge).
            \item Y tambien podemos saber (por el criterio de la integral) que para
            cualquiera $p > 1$ la serie converge.
        \end{itemize}


    % =====================================================
    % ========          SERIES TELESCOPICAS         =======
    % =====================================================
    \clearpage
    \section{Series Telescópicas}

        Las series telescópicas son muy lindas, para empezar lo que tenemos que
        hacer es ver que la Serie (Suma de todos los elementos de la Sucesión)
        tiene esta forma:

        \begin{equation}
            \sum_{n=1}^{\infty} b_n = (b_1-b_2) + (b_2-b_3) +  \cdots + (b_n-b_{n+1})
        \end{equation}

        O de manera mas concreta como:
        \begin{equation}
            \sum_{n=1}^{\infty} (b_{n} - b_{n+1})
        \end{equation}

        Y si te das cuenta todo eso se cancela, menos dos elementos, por lo podemos
        escribir así:
        \begin{equation}
            S_n = b_1 - b_{n+1}
        \end{equation}

        Y por lo tanto podemos ver que la serie (el límite de n en el
        infinito de las sumas parciales) es:

        \begin{equation}
            \sum_{n=1}^{\infty} b_n =  b_1 - \lim_{n \to \infty} b_{n+1}
        \end{equation}




    % =====================================================
    % ========          SERIES ALTERNANTES          =======
    % =====================================================
    \clearpage
    \section{Series Alternantes}

        Son un tipo de serie muy especial en la cual el signo cambia con cada
        termino. Las llamamos como serie alternante porque sus terminos alternan
        entre positivos y negativos.

        Podemos ver aquí que hay dos tipos de Series Alternantes:
        \begin{itemize}
            \item Si empezamos con números positivos es del tipo
                $\sum_{n=1}^{\infty} (-1)^{n-1} b_n$
            
            \item Si empezamos con números negativos es del tipo
            $\sum_{n=1}^{\infty} (-1)^n b_n$
        \end{itemize}

        Donde es bastante obvio que $b_n = |a_n|$


        Recuerda que nuestra serie es convergente si cumple con lo siguiente:
        \begin{itemize}
            \item $ 0 \leq b_{n+1} \leq b_n$
            \item $ \lim_{n \to \infty} b_n = 0$
        \end{itemize}

    \clearpage
    % ========   ESTIMACION DE ALTERNAS   =======
    \subsection{Estimación para Series Alternas}
        Una suma parcial de de cualquier serie convergente se puede usar como una
        aproximación a una suma total, pero no es muy utilizado, a menos que estime
        la exactitud de la aproximación.

        Esto es de verdad muy útil con las Series Alternantes, supongamos una Serie
        convergente, donde podemos escribir la Suma Parcial como $S = \sum (-1)^{n-1}b_n$
        que cumple con que:
        \begin{itemize}
            \item $ 0 \leq b_{n+1} \leq b_n$
            \item $ \lim_{n \to \infty} b_n = 0$
        \end{itemize}

        Entonces podemos decir que nuestra estimación será:
        \begin{equation}
            |R_n| = |S - S_n| \leq b_{n+1}
        \end{equation}

    % ========   CONVERGENCIA ABSOLUTA    =======
    \subsection{Convergencia Absoluta}

        Sea $\{a_n\}$ una sucesión:

        \begin{itemize}
            \item Decimos que la serie $\sum_{n=1}^{\infty} a_n$ es \emph{Absolutamente Convergente} si la serie $\sum_{n=1}^{\infty} |a_n|$ converge.

            \item Si la serie $\sum_{n=1}^{\infty} a_n$ converge pero la serie $\sum_{n=1}^{\infty} |a_n|$ diverge, decimos que la serie es \emph{Condicionalmente Convergente}.
        \end{itemize}

        Podemos crear un Teorema muy interesante:
        Si $\sum_{n=1}^{\infty} a_n$ es absolutamente convergente, entonces también es convergente.

        El Teorema anterior es muy útil, ya que garantiza que una serie absolutamente convergente es convergente.
        Sin embargo, su recíproco no es necesariamente cierto: Las series que son Convergentes pueden o no ser Absolutamente Convergentes. 

        El ejemplo más famoso es la serie cuyo $n$-ésimo término es $a_n=\dfrac{(-1)^{n-1}}{n}$, ya que $\sum_{n=1}^{\infty}a_n$ converge por el teorema anterior, pero $\sum_{n=1}^{\infty} |a_n| = \sum_{n=1}^{\infty} \frac{1}{n}$ diverge por el criterio de las Series P.



    % =====================================================
    % ========          SERIES DE POTENCIAS         =======
    % =====================================================
    \clearpage
    \section{Series de Potencias}

        Una serie de potencias es una serie donde $x$ es una variable
        y las $c_n$ son constantes que se denominan coeficientes de la serie.
        Para cada x establecida, la serie ya es una serie de constantes
        que puede probar para ver si son convergentes o divergentes.

        Repito, estas series son respecto a "dos variables":
        \begin{itemize}
            \item $x$ es variable totalmente libre, como una chica francesa.
            \item $c_n$ es un acrónimo para coloca aqui cualquier serie común, como el 
            $b_n$ de las alternas.
        \end{itemize}

        \begin{equation}
            \sum_{n=0}^{\infty} c_n x^n = c_0 + c_1x + c_2x^2 + c_3x^3 + \cdots
        \end{equation}

        Pero generalmente no es así como lo vemos, sino que tienen esta formula, 
        donde se le conoce como serie de pontencia centrada en a, en $(x-a)$ ó
        con respecto a a:

        \begin{equation}
            \sum_{n=0}^{\infty} c_n (x-a)^n = c_0 + c_1(x-a) + c_2(x-a)^2 + c_3(x-a)^3 + \cdots
        \end{equation}


        Aclaraciones de la Notación  
        \begin{itemize}
            \item Siempre que $x = a$ la serie va a converger
            \item Observe que al escribir el término correnpondiente a n=0 
                  en las ecuaciones 1 y 2, se ha adoptado la convención de
                  $(x-a)^0=1$ aun cuando $x=a$. 
        \end{itemize}


    % ========   RADIO DE CONVERGENCIA    =======
    \clearpage
    \subsection{Radio de Convergencia}
        Como puedes ver la $x$ en las series de potencias es una incognita que puede
        valer cualquier número, así decimos que el Radio de Convergencia es el conjunto de
        todos los valores de x tales que se cumple que dicha serie converge.

        Ahora veamos como sacar dicho intervalo - conjunto:

        \begin{itemize}
            \item Usa el Criterio de la Razón o de la Raíz como creas mas apropiado
            y despeja a $x$ de tu resultado, obtendras una desigualdad o algo parecido

            \item Ya casi terminas, lo único que te falta es ver que pasa cuando el 
            criterio que elegiste de 1 (pues recuerda que ambos criterios no te dicen nada
            si $L=1$), así que a patita verifica que pasa en ambos límites del intervalo para
            saber que pasa en ambos extremos (si son cerrados o abiertos). 
        \end{itemize}


        Obviamente para cualquier Serie de Potencias solo hay 3 posibilidades:
        \begin{itemize}
            \item Solo converge cuando $x-a$
            \item La serie converge siempre
            \item Existe un número R tal que la serie converge si $|x-a|<R$
        \end{itemize}

        Puede pasar en varios ejercicios que te den el rango de convergencia, así que quizá
        te resulte útil saber que para la serie:

        \begin{equation*}
            \sum_{k=0}^{\infty} x^k
        \end{equation*}

        Para lograr su radio tenemos que llegar a una expresión como: $\frac{1}{1-x}$
        donde sabemos que el radio es bien bien $|x|<1$

        % ==========================
        % ====    EJEMPLO  =========
        % ==========================
        \clearpage
        \subsubsection{Ejemplo 1}
            Un ejercicio muy sencillo es ver para que valores de $x$
            la siguiente serie es convergente:
            \begin{equation*}
               \sum_{n=1}^\infty \frac{(x-3)^n}{n}
            \end{equation*}

            Podemos encontrar la respuesta facilmente por el Criterio de la Razón como:
            \begin{equation*}
                \lim_{n \to \infty} \left| \frac{a_{n+1}}{a_n} \right| = 
                \lim_{n \to \infty} \left| \frac{\frac{(x-3)^{n+1}}{n+1}}{\frac{(x-3)^n}{n}} \right| = 
                \lim_{n \to \infty} \left| \frac{(x-3)^{n+1}}{n+1} \cdot \frac{n}{(x-3)^{n}} \right| =
                \lim_{n \to \infty} \frac{1}{1+\frac{1}{n}} |x-3| =
                |x-3|
            \end{equation*}

            Ahora como queremos cuando converge de verdad nos importa esto:
            \begin{equation*}
                |x-3| < 1 \to 2 < x < 4
            \end{equation*}

            Ahora ya solo checamos para 2 y para 4:

            Si podes el 4 en la serie se vuelve la serie $\Sigma \frac{1}{n}$ así es es obvio que diverge.

            Si pones 2, logramos llegar a la clasica $\Sigma \frac{(-1)^n}{n}$ que es una clasica
            alternante que converge.

            Por lo tanto ya para finalizar tenemos que nuestro radio de convergencia será:
            $2 \leq x < 4$








% ======================================================================================
% ===============================   CRITERIOS      =====================================
% ======================================================================================
\clearpage
\chapter{Criterios en Series}

    % =====================================================
    % ========       PRUEBA DE DIVERGENCIA        =========
    % =====================================================
    \clearpage
    \section{Prueba de la Divergencia}
        Esta es muy clásica y es muy fácil primero hacer esta antes
        de hacer nada más:

        \begin{itemize}
            \item \textbf{Original} Si la Serie $\sum_{n=1}^{\infty} a_n$ es
            convergente entonces $\lim_{n \to \infty} a_n = 0$

            \item \textbf{ContraPositiva} Si $\lim_{n \to \infty} a_n \neq 0$ entonces la Serie es Divergente.
        \end{itemize}

    % =====================================================
    % ========       PRUEBA DE LA INTEGRAL        =========
    % =====================================================
    \clearpage
    \section{Prueba de la Integral}
        Suponga que f es una función:

        \begin{itemize}
            \item Continua
            \item Positiva
            \item Decreciente en $[1, \infty)$
        \end{itemize}

        y sea $a_n = f(n)$

        Entonces este criterio nos dira que:
        \begin{itemize}
           \item Si $\int_1^{\infty}f(x) dx$ es convergente, entonces $\sum_{n=1}^{\infty} a_n$ es convergente
           \item Si $\int_1^{\infty}f(x) dx$ es divergente, entonces $\sum_{n=1}^{\infty} a_n$ es divergente
        \end{itemize}

        Cuando use la prueba de la integral no es necesario iniciar la serie o la integral en $n=1$.
        Asimismo, no es necesario que $f(x)$ sea siempre decreciente.
        Lo importante es que $f(x)$ sea decreciente por último, es decir, decreciente para x más
        grande que algún número N. 



    % =====================================================
    % ========   CRITERIO DE COMPARACION DIRECTA  =========
    % =====================================================
    \clearpage
    \section{Criterio de Comparación: Directa}

        Supón que $a _n > 0$ y que también $b_n > 0$. Osea que ambos terminos siempre seran positivos.
        Entonces:

        \begin{itemize}
            \item Si $\Sigma b_n$ es convergente y $a_n \leq b_n$, entonces $a_n$ es convergente. 
            \item Si $\Sigma b_n$ es divergente y $a_n \geq b_n$, entonces $a_n$ es divergente. 
        \end{itemize}

        Naturalmente, al usar la prueba por comparación es necesario tener alguna serie conocida $\Sigma b_n$
        para los fines de la comparación. La mayor parte de las veces se usan las series:

        \begin{itemize}
            \item Series P: $\Sigma \frac{1}{n^p}$ que convergen si $p>1$ y divergen si $p\leq 1$
            \item Series Geométricas: $\Sigma ar^{n-1}$ que convergen si $|r|<1$ y divergen si $|r|\geq 1$
        \end{itemize}

        La condición $a_n \leq b_n$ o bien, $a_n \geq b_n$ de la prueba por comparación es para toda n, es
        necesario comprobar sólo que se cumple para $n \geq N$, donde N es un entero establecido, porque
        la convergencia de una serie no está afectada por un número finito de términos.

        % ==========================
        % ====    EJEMPLO  =========
        % ==========================
            \subsubsection{Ejemplo 1}
            Busquemos si la siguiente Serie diverge o converge:

            \begin{equation*}
                \sum_{n=1}^{\infty} \frac{5}{2n^2 +4n +3}
            \end{equation*}

            Ahora apliquemos el criterio de comparación: Podemos ver que esta serie se pacere mucho a esta
            que ya conocemos todos, a esta serie de ayuda la llamaremos $\Sigma b_n$:

            \begin{equation*}
                \sum b_n = \sum_{n=1}^{\infty} \frac{5}{2n^2} = \frac{5}{2}\sum_{n=1}^{\infty} \frac{1}{n^2}
            \end{equation*}

            Bueno, podemos decir que:
            \begin{equation*}
                \frac{5}{2n^2 +4n +3} < \frac{5}{2n^2}
            \end{equation*}
            Simplemente por el denominador.

            Y veamos que todo se cumplio, ademas sabemos que la serie $\Sigma \frac{1}{n^2}$ es convergente,
            entonces es seguro que la serie original que teniamos tambien lo sea. :D



    % =====================================================
    % ========   CRITERIO DE COMPARACION LIMITE   =========
    % =====================================================
    \clearpage
    \section{Criterio de Comparación: Limites}

        Supón que $a _n > 0$ y que también $b_n > 0$. Osea que ambos terminos siempre seran positivos.

        Entonces si:
        $\lim_{n \to \infty} \left( \frac{a_n}{b_n} \right) = L$

        (Donde obviamente L debe ser positivo y finito)

        Si todo esto se cumple entonces alguna de las dos proposiciones deben ser verdad:
        \begin{itemize}
            \item Ambas $\Sigma a_n$ y $\Sigma b_n$ divergen.
            \item Ambas $\Sigma a_n$ y $\Sigma b_n$ convergen.
        \end{itemize}

        % ==========================
        % ====    EJEMPLO  =========
        % ==========================
        \subsubsection{Ejemplo 1}
            Busquemos si la siguiente Serie diverge o converge:

            \begin{equation*}
                \sum_{n=1}^{\infty} \frac{3n^2+2}{(n^2-5)^2}
            \end{equation*}

            Antes que hacer nada, lo mejor es expandir:
            \begin{equation*}
                \sum_{n=1}^{\infty} \frac{3n^2+2}{n^4-10n^2+25}
            \end{equation*}
             
            Antes que seguir a nada, vemos si con la prueba de la divergencia podemos mostrar que diverge
            (para ahorrar trabajo):
            \begin{equation*}
                \lim_{n \to \infty} \frac{3n^2+2}{n^4-10n^2+25} = ?
            \end{equation*}

            Esto lo podemos calcular de muchas maneras, por ejemplo:
            \begin{equation*}
                \lim_{n \to \infty} \frac{ \frac{3n^2}{n^4} +\frac{2}{n^4}  }{ 1 - \frac{10n^2}{n^4} + \frac{25}{n^4} } = \frac{0}{1+0} = 0
            \end{equation*}

            Ok, paso esa prueba, lamentablemente esto no es
            suficiente para probar que converge.
            Ahora apliquemos el criterio de comparación: Podemos ver que esta serie se pacere mucho a esta que ya conocemos todos, a esta serie de ayuda la llamaremos $\Sigma b_n$:

            \begin{equation*}
                \sum b_n = \sum_{n=1}^{\infty} \frac{n^2}{n^4} = \sum_{n=1}^{\infty} \frac{1}{n^2}
            \end{equation*}

            Ahora aplicando lo que acabamos de ver:
            \begin{equation*}
                \lim_{n \to \infty} \left( \frac{a_n}{b_n} \right) = \left( \frac{ \frac{3n^2+2}{n^4-10n^2+25} }{ \frac{1}{n^2} } \right) =  \left( \frac{3n^4+2n^2}{n^4-10n^2+25} \right) = 3
            \end{equation*}

            Y veamos que todo se cumplio, 3 es finito y positivo y sabemos que la serie $\Sigma \frac{1}{n^2}$ es convergente, entonces es seguro que la serie original que teniamos tambien lo sea. :D


        % ==========================
        % ====    EJEMPLO  =========
        % ==========================
        \subsubsection{Ejemplo 2}
            Busquemos si la siguiente Serie diverge o converge:

            \begin{equation*}
                \sum_{n=1}^{\infty} \frac{n^{k-1}}{n^k+7}
            \end{equation*}
             
            Antes que seguir a nada, vemos si con la prueba de la divergencia podemos mostrar que diverge (para ahorrar trabajo):
            \begin{equation*}
                \lim_{n \to \infty} \frac{n^{k-1}}{n^k+7} = ?
            \end{equation*}

            Esto lo podemos calcular de muchas maneras, por ejemplo:
            \begin{equation*}
                \lim_{n \to \infty} \frac{n^{k-1}}{n^k+7} = \frac{ \frac{n^{k-1}}{n^k} }{1+\frac{7}{n^k}} = \frac{0}{1+0} = 0
            \end{equation*}

            Ok, paso esa prueba, lamentablemente esto no es suficiente para probar que converge, es más parece que debería diverger, así que probemos para eso:

            Ahora apliquemos el criterio de comparación, podemos ver que esta serie se pacere mucho a esta que ya conocemos todos, a esta serie de ayuda la llamaremos $\Sigma b_n$:

            \begin{equation*}
                \sum b_n = \sum_{n=1}^{\infty} \frac{n^{k-1}}{n^k} = \sum_{n=1}^{\infty} \frac{1}{n}
            \end{equation*}

            Sabemos que esta serie diverge.

            Ahora aplicando lo que acabamos de ver:
            \begin{equation*}
                \lim_{n \to \infty} \left( \frac{a_n}{b_n} \right) = \left( \frac{ \frac{n^{k-1}}{n^k+7} }{ \frac{1}{n} } \right) =  \left( \frac{n^k}{n^k+7} \right) = 1
            \end{equation*}

            Y veamos que todo se cumplio, 1 es finito y positivo y sabemos que la serie $\Sigma \frac{1}{n}$ es divergente, entonces es seguro que la serie original que teniamos tambien lo es :D


        % ==========================
        % ====    EJEMPLO  =========
        % ==========================
        \subsubsection{Ejemplo 3}
            Busquemos si la siguiente Serie diverge o converge:

            \begin{equation*}
                \sum_{n=1}^{\infty} \frac{3^n+2}{4^n-1}
            \end{equation*}

            Antes que seguir a nada, vemos si con la prueba de la divergencia podemos mostrar que diverge (para ahorrar trabajo). Esto se calcula muy facilmente porque el demoni- nador crece mucho mas rapidamente

            \begin{equation*}
                \lim_{n \to \infty}  \frac{3^n+2}{4^n-1} = 0
            \end{equation*}


            Ok, paso esa prueba, lamentablemente esto no es
            suficiente para probar que converge.
            Ahora apliquemos el criterio de comparación: Podemos ver que esta serie se pacere mucho a esta que ya conocemos todos, a esta serie de ayuda la llamaremos $\Sigma b_n$:

            \begin{equation*}
                \sum b_n = \sum_{n=1}^{\infty} \frac{3^n}{4^n} = \sum_{n=1}^{\infty} \left( \frac{3}{4} \right)^n 
            \end{equation*}

            Esto es una serie geometrica que converge, pues $|r| < 1$

            Ahora aplicando lo que acabamos de ver:
            \begin{equation*}
                \lim_{n \to \infty} \left( \frac{a_n}{b_n} \right) = \left( \frac{ \frac{3^n+2}{4^n-1} }{ \left( \frac{3}{4} \right)^n  } \right) =  \left( \frac{12^n + 2 \cdot 4^n}{12^n -3^n} \right) = \left( \frac{1 + 2 (\frac{1}{3})^n}{1 +  (\frac{1}{4})^n} \right) = 1
            \end{equation*}

            Y veamos que todo se cumplio, 1 es finito y positivo y sabemos que la serie $\Sigma (\frac{3}{4})^n$ es convergente, entonces es seguro que la serie original que teniamos tambien lo sea :D


    % =====================================================
    % ========       CRITERIO DE LA RAZON           =======
    % =====================================================
    \clearpage
    \section{Criterio de la Razón}

        Sea una $\Sigma a_n$ una series de términos positivos, tal que:

        \begin{equation}
            \lim_{n \to \infty} \left| \frac{a_{n+1}}{a_n} \right| = L
        \end{equation}

        Entonces:
        \begin{itemize}
            \item $L < 1$ : La Serie es absolutamente convergente.
            \item $L > 1$ ó $L = \infty$ : La Serie diverge.
            \item $L = 1$ : No nos dirá nada (por ejemplo cualquier serie P nos dará 1)
        \end{itemize}

        Pero si que podemos llegar a algo más:
        Si L da uno, podemos aplicar L' Hopital y volver a comprobar:

        \begin{equation}
        \left| \frac{\frac{d}{dn} (a_{n+1})}{\frac{d}{dn} (a_n)} \right| = L
        \end{equation}



    % =====================================================
    % ========       CRITERIO DE LA RAIZ            =======
    % =====================================================
    \clearpage
    \section{Criterio de la Raíz}

        Considera a este como el hermano perdido de la comprobación por razón.
        Es conveniente aplicar la siguiente prueba cuando hay potencias n-ésimas.
        Su demostración es similar a la de la prueba de la razón. 

        Si:

        \begin{equation}
            \lim_{n \to \infty} \sqrt[n]{|a_n|} = L
        \end{equation}

        Entonces:
        \begin{itemize}
            \item $L < 1$ : La Se es absolutamente convergente.
            \item $L > 1$ ó $L = \infty$ : La Serie diverge.
            \item $L = 1$ : No nos dirá nada, digamos.
        \end{itemize}

        ¿Ves?, te dije que se parecia mucho a la de razón.

        % ==========================
        % ====    EJEMPLO  =========
        % ==========================
        \subsubsection{Ejemplo 1}
            Busquemos si la siguiente Serie diverge o converge:

            \begin{equation*}
                \sum_{n=1}^{\infty} \left( \frac{2n+3}{3n+2} \right)^n
            \end{equation*}

            Probemos entonces la Raíz:
            \begin{equation*}
                 \lim_{n \to \infty} \sqrt[n]{ \left( \frac{2n+3}{3n+2} \right)^n } =
                 \lim_{n \to \infty} \frac{2n+3}{3n+2} = \frac{2}{3}
            \end{equation*}

            Y como $\frac{2}{3}$ es menor que 1 sabemos que nuestra serie converge.


    % =====================================================
    % ========       CRITERIO DE LAS ALTERNANTES    =======
    % =====================================================
    \clearpage
    \section{Criterio de las Series Alternantes}

        Para probar que una Serie Alternante $\sum_{n=1}^{\infty} (-1)^{n-1} b_n$ y $\sum_{n=1}^{\infty} (-1)^n b_n$ es convergente entonces tendrá que cumplir que:

        \begin{itemize}
            \item $\{b_n\}$ es una sucesión decreciente, es decir, $b_n \geq b_{n+1}$ para $n$ suficientemente grande
            \item Que el $\lim_{n \to \infty} b_n = 0$
        \end{itemize}

        Una observación es que este criterio solo sirve para demostrar convergencia, es decir, si alguna de las dos condiciones no se cumple sobre la serie alternante, no podemos concluir nada y será necesario usar otro criterio.

        % ==========================
        % ====    EJEMPLO  =========
        % ==========================
        \subsubsection{Ejemplo 1}
            Una sencilla para encaminarnos:
            \begin{equation*}
                \sum_{n=1}^{\infty} (-1)^{n-1} \frac{1}{n}
            \end{equation*}


             \begin{itemize}
                \item Paso 1: Limite $\lim_{n \to \infty} \frac{1}{n}=0$
                \item Paso 2: ¿Es Decreciente? Es decir :$\frac{1}{n}-\frac{1}{n+1} \geq 0 $
             \end{itemize}

            Como es verdadero entonces esta suma es convergente.



    % =====================================================
    % ========    ESTRATEGIAS PARA USAR CRITERIOS   =======
    % =====================================================
    \clearpage
    \section{Estrategias para Usar Criterios}

    \begin{itemize}
        \item (Prueba de Divergencia) Verifica que el n-ésimo sea 0 cuando n tienda a infinito.
        \item Verifica si la que tienes es una Serie P ó Geométrica, si si ya sabes que hacer ;)
        \item (Comparación) Si se pacere a una Serie P o Geométrica, usa alguna de las de Comparación.
        \item (Convergencia Absoluta) Si quitando que sea alternante se vuelve algo como una P o Geométrica, intenta convergencia absoluta.
        \item (Criterio de Alternantes) Vea que si es alternante.
        \item (Razón) Si tienes Factoriales o Potencias, prueba con Razón.
        \item (Raíz) Si nada funciona, o tienes un termino elevado a la n, intenta la Raíz.
        \item (Integrales) Si estás re muerto, intenta con Integrales.
    \end{itemize}



% =====================================================
% ============        BIBLIOGRAPHY   ==================
% =====================================================
\clearpage
\bibliographystyle{plain}
    \begin{thebibliography}{9}

    % ============ REFERENCE #1 ========
    \bibitem{Sitio1} 
        ProbRob
        \\\texttt{Youtube.com}


     

\end{thebibliography}



\end{document}